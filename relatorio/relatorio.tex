\documentclass[12pt]{article}
\usepackage[utf8]{inputenc}
\usepackage[portuguese]{babel}
\usepackage{geometry}
\geometry{a4paper, margin=1in}
\usepackage{graphicx}
\usepackage{hyperref}
\usepackage{tabularx} % Para formatar os autores de forma mais elegante

\title{
    \vspace{2in}
    \textbf{Trabalho Prático 1 - Linguagens de Programação}\\
    \vspace{0.5in}
    \large DCC024 - Linguagem de Programação - 2024.1\\
    \vspace{0.25in}
    \large Professor: Marcio Costa Santos\\
    \vspace{0.25in}
    \large Linguagem sorteada: Lua\\
    \vspace{1.5in}
}

\author{%
\begin{tabularx}{\textwidth}{@{}lX@{}}
\textbf{Integrantes do Grupo:} & \\
Daniel Oliveira Barbosa & Matrícula\\
Diane Fenzi Goncalves & Matrícula\\
Frederico Dolher Souza Baker & 2021031513\\
Rubia Alice Moreira de Souza & Matrícula\\
Victor Yuji Yano & Matrícula\\
Wallace Eduardo Pereira & Matrícula\\
\end{tabularx}\\[1in]
}

\date{Data de Entrega: 20/05/2024}

\begin{document}

\begin{titlepage}
\maketitle
\thispagestyle{empty}
\end{titlepage}

\newpage
\tableofcontents
\thispagestyle{empty}
\clearpage

\setcounter{page}{1}

\section{Introdução}
Este relatório tem como objetivo apresentar os resultados do estudo sobre a linguagem de programação Lua, conforme solicitado no Trabalho Prático 1 da disciplina DCC024 - Linguagem de Programação, ministrada pelo professor Marcio Costa Santos. A linguagem Lua foi sorteada para nosso grupo, e neste documento, exploraremos seu histórico, características principais, análise de usabilidade, eficiência, pontos fortes e pontos fracos.

\section{Histórico da Linguagem Lua}
Aqui, oferecemos uma visão geral sobre o surgimento da Lua, seu desenvolvimento ao longo dos anos e como ela se posicionou no mundo da programação, destacando sua origem brasileira e sua importância no cenário global.

\section{Características Básicas da Linguagem}
Detalhamos as características fundamentais da Lua, abordando sua sintaxe, tipagem dinâmica, uso como linguagem de script, e sua extensibilidade, que tem sido um dos principais atrativos da linguagem.

\section{Análise de Usabilidade e Eficiência}
Nesta seção, avaliamos a Lua em termos de facilidade de aprendizado e uso, desempenho em execução, e como ela se compara com outras linguagens populares em termos de eficiência.

\section{Pontos Fortes e Fracos}
Exploramos os principais aspectos que fazem da Lua uma escolha preferida para determinados projetos, incluindo sua portabilidade, leveza e flexibilidade.


\section{Conclusão}
Concluímos nosso relatório com uma síntese das descobertas sobre a linguagem Lua, refletindo sobre sua posição no panorama atual das linguagens de programação e suas potenciais aplicações futuras.

\section{Referências}
Listamos todas as fontes utilizadas na preparação deste relatório, garantindo que todas as informações apresentadas sejam verificáveis e baseadas em conhecimento confiável.

\end{document}
